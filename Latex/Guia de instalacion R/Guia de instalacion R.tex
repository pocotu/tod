\documentclass[12pt,hidelinks]{article}


\begin{document}

\title{Guia de instalacion de R}

\maketitle

\section{Comprobacion de software de su computadora}

\begin{itemize}
\item MAC: haz clic en la manzana en la parte superior izquierda de la pantalla y elige 'Acerca de esta Mac'. El último software hasta la fecha es Mac OS Big Sur. 
\item WINDOWS: Windows: probablemente tengas Windows 8, 10 O 11.Para obtener más información sobre Windows 10, haga clic en el botón de inicio y en el ícono de Configuración. Luego, haga clic en Sistema y luego en 'Acerca de' en el panel izquierdo. Le dará información como si tiene un procesador de 32 o 64 bits (que es útil saber para una instalación posterior).
\end{itemize}

\section{Instalacion de R}
\begin{itemize}

\item Siga este enlace: https://cran.r-project.org/mirrors.html (Escoja la region segun donde se encuentra)
\item Para Mac: Para Mac, elija el archivo que corresponda a la versión de su sistema Mac OS X. Si tiene todo actualizado, puede descargar e instalar 4.0.3 (simplemente haga clic y abra el archivo .pkg).
\item Para Windows: haga clic en 'instalar R por primera vez', que es azul y en negrita junto a 'base'. Luego 'Descargar R para Windows'. Si descarga e instala con los valores predeterminados, instalará las versiones R de 32 y 64 bits. Puede eliminar el que no coincida con su procesador (consulte más arriba cómo verificar esa información).

\end{itemize}

Si tienen dudas consultar con el autor

\end{document}

